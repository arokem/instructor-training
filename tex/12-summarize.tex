\begin{center}
\rule{3in}{0.4pt}
\end{center}
title: ``Afternoon Wrap-Up''
teaching: 5
exercises: 15
questions:
- ``What have we learned?''
objectives:
- ``Understand overnight homework.''
- ``Explain pros and cons of `one up, one down' feedback.''
keypoints:
- ``Use `one up, one down' to get wide-ranging feedback.''
---

\begin{quotation}   %  class="challenge"
\subsection*{Reflecting on the Day}

Before we wrap up for the day, take a moment to think over 
everything we covered today.  On a piece of paper, write 
down something that captures what you want to remember about 
the day.  The trainers won't look at this - it's just for you.

If you don't know where to start, consider 
the following list for a starting point:

\begin{itemize}
\item draw a concept map, connecting the material
\item draw pictures or a comic depicting one of the day's concepts
\item write an outline of the topics we covered
\item write a paragraph or “journal” entry about your 
exeperience of the training today
\item write down one thing that struck you the most
\end{itemize}
\end{quotation}   %  class="challenge"

As homework overnight,
please read our \href{\{\{ site.swc\_site \}\}/workshops/operations/}{operations guide}
and the checklists it links to.
These recommendations and how-to guides summarize what we have learned
(often the hard way)
about organizing and running workshops.
When you arrive tomorrow,
we will ask you to add one question about our operations to a list;
we will then do our best to answer all of those questions during the day.

Please also read ``\href{\{\{ page.root \}\}/files/papers/wilson-lessons-learned-2016.pdf}{Software Carpentry: Lessons Learned}'',
which summarizes what we have learned over 18 years of teaching basic computing skills to researchers.

\begin{quotation}   %  class="callout"
\subsection*{End-of-Day Feedback}

We frequently ask workshop participants to give us feedback at the end
of each day using a technique called ``one up, one down''.  The
instructor asks the learners to alternately give one positive and one
negative point about the day, without repeating anything that has
already been said.  This requirement forces people to say things they
otherwise might not: once all the ``safe'' feedback has been given,
participants will start saying what they really think. The instructor
writes down the feedback in the Etherpad or a text editor,
but is not supposed to comment on the feedback while collecting it.

When called upon by the instructor, add one positive or one negative
point to the growing list \emph{without} repeating anything that has
already been said.
\end{quotation}   %  class="callout"
