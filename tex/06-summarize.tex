\begin{center}
\rule{3in}{0.4pt}
\end{center}
title: ``Morning Wrap-Up''
teaching: 5
exercises: 10
questions:
- ``What have we learned?''
objectives:
- ``Explain pros and cons of minute cards as a feedback mechanism.''
- ``Summarize morning's lessons.''
keypoints:
- ``Have learners write minute cards as exit tickets to get actionable feedback.''
---

We frequently use sticky notes as \emph{minute cards}: before each break,
learners take a minute to write one positive thing on the green sticky
note (e.g., one thing they've learned that they think will be useful),
and one thing they found too fast, too slow, confusing, or irrelevant
on the red one.  They can use the red sticky note for questions that
haven't yet been answered.  While they are enjoying their coffee or
lunch, the instructors review and cluster these to find patterns.  It
only takes a few minutes to see what learners are enjoying, what they
still find confusing, what problems they're having, and what questions
are still unanswered.

\begin{quotation}   %  class="challenge"
\subsection*{Fill In Minute Cards}

Write one thing you learned this morning that you found useful on
your green sticky note, and one question you have about the material
on the red.  Do \emph{not} put your name on the notes: this is meant to
be anonymous feedback.  Add your notes to the pile by the door as
you leave to get coffee.
\end{quotation}   %  class="challenge"
