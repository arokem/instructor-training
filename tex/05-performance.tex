\begin{center}
\rule{3in}{0.4pt}
\end{center}
title: ``Teaching as a Performance Art''
teaching: 15
exercises: 45
questions:
- ``How do teachers improve?''
objectives:
- ``Explain why standardized testing cannot by itself improve educational outcomes, and why and how peer-to-peer lesson study can.''
- ``Explain how teaching practices are actually transferred and adopted.''
- ``Summarize the strengths and weaknesses of your teaching practices.''
keypoints:
- ``Great teachers are made, not born.''
- ``Formal written descriptions of teaching practices are ineffective.''
- ``Lesson study (`jugyokenkyu') is essential to transferring skills between teachers.''
- ``Feedback is most effective when those involved share ground rules and expectations.''
---
Continuing with our theme of instructor-as-performer,
let's discuss some research that suggests that improvement in educational outcomes
must be driven by practice and communities of practice.
The tricks and techniques that make teaching effective cannot be mandated from above,
but will develop into widespread effectiveness
only as they are shared through a community of teaching practice.

\subsection*{Lesson Study}

Many people assume that teachers are born, not made.
From politicians to researchers and teachers themselves,
most reformers have designed systems to find and promote those who can teach
and eliminate those who can't.
As Elizabeth Green describes in
\emph{\href{http://www.amazon.com/Building-Better-Teacher-Teaching-Everyone/dp/0393081591/}{Building a Better Teacher}},
though,
that assumption is wrong,
which is why educational reforms based on it have repeatedly failed.

The book is written as a history of the people who have put that puzzle together in the US.
Its core begins with a discussion of what James Stigler discovered during a visit to Japan in the early 1990s:

\begin{quotation}   %  class="quotation"
Some American teachers called their pattern ``I, We, You'':
After checking homework,
teachers announced the day's topic,
demonstrating a new procedure (I)\ldots{}
Then they led the class in trying out a sample problem together (We)\ldots{}
Finally, they let students work through similar problems on their own,
usually by silently making their way through a worksheet (You)\ldots{}

The Japanese teachers, meanwhile, turned ``I, We, You'' inside out.
You might call their version ``You, Y'all, We.''
They began not with an
introduction,
but a single problem that students spent ten or twenty
minutes working through alone (You)\ldots{}
While the students worked,
the teacher wove through the students' desks,
studying what they came up with and taking notes to remember who had which idea.
Sometimes the teacher then deployed the students to discuss the problem in small
groups (Y'all).
Next, the teacher brought them back to the whole group,
asking students to present their different ideas for how to solve the problem on the chalkboard\ldots{}
Finally, the teacher led a discussion,
guiding students to a shared conclusion (We).
\end{quotation}   %  class="quotation"

It's tempting but wrong to think that this particular teaching technique is Japan's secret sauce.
The actual key is revealed in the description of Akihiko Takahashi's work.
In 1991,
he visited the United States in a vain attempt to find the classrooms described a decade earlier
in a report by the National Council of Teachers of Mathematics.
He couldn't find them.
Instead, he found that American teachers met once a year (if that) to exchange ideas about teaching,
compared to the weekly or even daily meetings he was used to.
What was worse:

\begin{quotation}   %  class="quotation"
The teachers described lessons they gave and things students said,
but they did not \emph{see} the practices.
When it came to observing actual lessons---watching each other teach---they
simply had no opportunity\ldots{}
They had, he realized, no \emph{jugyokenkyu}.
Translated literally as ``lesson study'',
\emph{jugyokenkyu} is a bucket of practices that Japanese teachers use to hone their craft,
from observing each other at work to discussing the lesson afterward
to studying curriculum materials with colleagues.
The practice is so pervasive in Japanese schools that it is\ldots{}effectively invisible.

And here lay the answer to [Akihiko's] puzzle.
Of course the American teachers' work fell short of the model set by their best thinkers\ldots{}
Without \emph{jugyokenkyu}, his own classes would have been equally drab.
Without \emph{jugyokenkyu}, how could you even teach?
\end{quotation}   %  class="quotation"

So what does \emph{jugyokenkyu} look like in practice?

\begin{quotation}   %  class="quotation"
In order to graduate,
education majors not only had to watch their assigned master teacher work,
they had to effectively replace him,
installing themselves in his classroom first as observers and then,
by the third week, as a wobbly\ldots{}approximation of the teacher himself.
It worked like a kind of teaching relay.
Each trainee took a subject,
planning five days' worth of lessons\ldots{} [and then] each took a day.
To pass the baton,
you had to teach a day's lesson in every single subject:
the one you planned and the four you did not\ldots{}
and you had to do it right under your master teacher's nose.
Afterward,
everyone---the teacher, the college students, and sometimes even another outside observer---would
sit around a formal table to talk about what they saw.

[Trainees] stayed in\ldots{}class until the students left at 3:00 pm,
and they didn't leave the school until they'd finished discussing the day's events,
usually around eight o'clock.
They talked about what [the master teacher] had done,
but they spent more time poring over how the students had responded:
what they wrote in their notes;
the ideas they came up with, right and wrong;
the architecture of the group discussion.
The rest of the night was devoted to planning\ldots{}

\ldots{}By the time he arrived in [the US],
[Akihiko had] become\ldots{}famous\ldots{}
giving public lessons that attracted hundreds,
and, in one case, an audience of a thousand.
He had a seemingly magical effect on children\ldots{}
But Akihiko knew he was no virtuoso.
``It is not only me,'' he always said\ldots{}
``\emph{Many} people.''
After all, it was his mentor\ldots{}who had taught him the new approach to teaching\ldots{}
And [he] had crafted the approach along with the other math teachers
in [his ward] and beyond.
Together, the group met regularly to discuss their plans for teaching\ldots{}
[At] the end of a discussion,
they'd invite each other to their classrooms to study the results.
In retrospect,
this was the most important lesson:
not how to give a lesson, but how to study teaching,
using the cycle of \emph{jugyokenkyu} to put\ldots{}work under a microscope and improve it.
\end{quotation}   %  class="quotation"

Putting work under a microscope in order to improve it is commonplace in sports and music.
A professional musician, for example,
will dissect half a dozen different recordings of ``Body and Soul'' or ``Smells Like Teen Spirit'' before performing it.
They would also expect to get feedback from fellow musicians during practice and after performances.
Many other disciplines work this way too:
the Japanese drew inspiration from \href{https://en.wikipedia.org/wiki/W.\_Edwards\_Deming}{Deming}'s ideas
on continuous improvement in manufacturing,
while the adoption of code review over the last 15 years
has done more to improve everyday programming than any number of books or websites.

But this kind of feedback isn't part of teaching culture in English-language school systems:
in the US, Canada, the UK, Australia, and elsewhere,
what happens in the classroom stays in the classroom.
Teachers don't watch each other's lessons on a regular basis,
so they can't borrow each other's good ideas.
The result is that \emph{every teacher has to invent teaching on their own}.
They may get lesson plans and assignments from colleagues,
the school board,
a textbook publisher,
or the Internet,
but each teacher has to figure out on their own how to combine that with
the theory they've learned in education school
to deliver an actual lesson in an actual classroom for actual students.

Fincher and her colleagues studied how teaching practices are actually transferred
using both \href{\{\{ page.root \}\}/files/papers/fincher-warrens-questions-2007.pdf}{a detailed case study}
and
\href{\{\{ page.root \}\}/files/papers/fincher-stories-change-2012.pdf}{analysis of change stories}.
The abstract of the latter paper sums up their findings:

\begin{quote}   %  class="quotation"
Innovative tools and teaching practices often fail to be adopted by educators in the field,
despite evidence of their effectiveness.
Naïve models of educational change assume this lack of adoption arises from
failure to properly disseminate promising work,
but evidence suggests that dissemination via publication is simply not effective\ldots{}
We asked educators to describe changes they had made to their teaching practice
and analyzed the resulting stories\ldots{}
Of the 99 change stories analyzed,
only three demonstrate an active search for new practices or materials on the part of teachers,
and published materials were consulted in just eight of the stories.
Most of the changes occurred locally,
without input from outside sources,
or involved only personal interaction with other educators.
\end{quote}   %  class="quotation"

As reported in \href{\{\{ page.root \}\}/files/papers/barker-practice-adoption-2015.pdf}{this paper},
Barker et al found something similar in 2015:

\begin{quote}   %  class="quotation"
Adoption is not a ``rational action,'' however, but an iterative
series of decisions made in a social context, relying on normative
traditions, social cueing, and emotional or intuitive processes\ldots{}
[F]aculty are not likely to use educational research findings as the
basis for adoption decisions. Faculty become aware of innovative
practices either because a problem leads them to intentionally seek
them out, or they hear about them through funded initiatives,
conferences and journals, or from colleagues. They experiment (or
not) for several reasons, depending on institutional expectations
and policies, perceived costs and benefits for themselves and
students, and the influence of role models. Faculty tend to trust
other faculty whose work and institutional context is more like
their own. The choice to try out practices competes with the need to
``cover'' material, as well as with classroom layouts. Positive
student feedback is taken as strong evidence by faculty that they
should continue a practice.
\end{quote}   %  class="quotation"

\begin{quotation}   %  class="callout"
\subsection*{Learning Sideways}

The phrase \emph{lateral knowledge transfer} is sometimes used to describe what happens
when someone intended to teach one thing,
but their audience learned another along the way.
For example,
an instructor might set out to show people how to do a particular statistical analysis in R,
but what her learners might take away is some new keyboard shortcuts in R Studio.
Live coding makes this much more likely
because it allows learners to see the ``how'' as well as the ``what''.
\end{quotation}   %  class="callout"

\begin{quotation}   %  class="challenge"
\subsection*{Giving Feedback}

Watch \href{https://www.youtube.com/watch?v=-ApVt04rB4U}{this video} as a group
and then give feedback on it.
Try to organize feedback along two axes:
positive vs. negative
and content (what was said) vs. presentation (how it was said).
\end{quotation}   %  class="challenge"

\begin{quotation}   %  class="challenge"
\subsection*{Feedback on Yourself}

\begin{enumerate}
\item Split into groups of three
\item Have each person introduce themselves
and then explain, in no more than 90 seconds,
the key idea or ideas from
the Carpentry lesson episode they chose before the start of the training course
to another person in the group
while the third person records it (video and audio)
using a cell phone or some other handheld device.
\item After the first person finishes,
rotate roles
(she becomes the videographer,
her audience becomes the instructor,
the person who was recording becomes the audience)
and then rotate roles again.
\item After everyone in the group of three has finished teaching,
watch the videos as a group.
Everyone gives feedback on all three videos,
i.e., people give feedback on themselves as well as on others.
\item After everyone has given feedback on all of the videos,
return to the main group and put everyone's feedback about you into the Etherpad.
\end{enumerate}
\end{quotation}   %  class="challenge"

\subsection*{On Stage}

\subsection*{What Are Your Tells?}

How was the experience of being videoed/receiving feedback?  What
did people notice?  What are some of your ``tells''?


Everyone has nervous habits.  While these habits are often not as noticeable
as you would think, it's good to identify ways to keep yourself from pacing,
or fiddling with your jewellery, or not looking at the audience.
For example, many of us become ``Mickey Mouse'' versions of ourselves when we're nervous,
i.e., we talk more rapidly than usual, in a higher-pitched voice,
and wave our arms around more than we usually would.

But just like everyone has their own nervous habit, each person has their own
strengths.  Musicians can be very different (but equally effective!)
in their performance of the same piece; similarly, teachers can present the same material
in very different ways.  The uniqueness of your teaching style can and should
be based on your strengths.  This is why it's just as important to identify strengths
as weaknesses when trying to improve your teaching.  It's good to know what you do well!

\subsection*{Feedback}

Sometimes it can be hard to receive feedback, especially negative feedback.

Feedback is most effective when the people involved
can share ground rules and expectations.  This is especially important
when the instructor and/or students have different cultural or
domain expectations about feedback.

Here is a list of different
ways that you, as the instructor, can set the stage for receiving feedback
in a way that helps you improve:

\begin{itemize}
\item Initiate feedback.  It's better to ask for feedback than to receive it unwillingly.
\item Choose your own questions and ask for specific feedback.  For example:

\begin{itemize}
\item ``What is one thing I could have done as an instructor to make this
lesson more effective?''
\item ``If you could pick one thing from the lesson to go over again, what
would it be?''
\end{itemize}
\end{itemize}

Specific feedback like this is more useful than a generic ``that was great!'' or ``that
was terrible!''  Also, writing your own feedback questions allows you to frame
feedback in a way that is helpful to you - the questions above
reveal what didn't work in your teaching, but
read as professional suggestions rather than personal judgments.

\begin{itemize}
\item Communicate expectations. If your teaching feedback is taking the form of an
observation (and you're comfortable enough with the observer), tell
that person how they can best communicate their feedback to you.
\item Balance positive and negative feedback.


\begin{itemize}
\item Ask for or give ``compliment sandwiches'' (one positive, one negative, one positive)
\item Ask for both types of feedback
\end{itemize}
\item Use a feedback translator.  Have a fellow
instructor (or other trusted person in the room) read over all the feedback and give
an executive summary.   It can be easier to hear ``It sounds like most people
are following, so you could speed up'' than to read several notes all saying, ``this is
too slow'' or ``this is boring''.
\end{itemize}

This is part of the reason for Data Carpentry and Software Carpentry's rule,
``Never teach alone.''
Having another instructor in the classroom saves your voice
(it's hard to talk for two days straight),
but more importantly,
it's a chance for instructors to learn from one another and be a supportive voice
in the room.

Finally, be kind to yourself.
Mental habits matter: if you're a self-critical person,
it's OK to remind yourself that:

\begin{itemize}
\item It's not personal.
\item Look at the positives along with the negatives.
\item Etc.
\end{itemize}

\begin{quotation}   %  class="challenge"
\subsection*{Feedback on Feedback}

Watch either \href{https://vimeo.com/139316669}{this video} (8:40)
or \href{https://vimeo.com/139181120}{this one} (11:42).
Take notes about the presentation,
and divide those into four groups
based on whether they are positive or negative
and whether they are about the content (what was said)
or the presentation (how it was said, e.g., body language).
Compare your notes with those made by other people,
and with the feedback given by your instructor.
\end{quotation}   %  class="challenge"

\begin{quotation}   %  class="challenge"
\subsection*{Feedback on Yourself, Part II}

Later in the training,
repeat the first challenge exercise; however, when it comes time to give feedback,
use the same 2x2 scheme in the previous challenge.
\end{quotation}   %  class="challenge"

\begin{quotation}   %  class="challenge"
\subsection*{Learn More About Feedback}

Read Gormally et al's ``\href{\{\{ page.root \}\}/files/papers/gormally-teaching-feedback-2014.pdf}{Feedback about Teaching in Higher Ed}''
and discuss ways you could make peer-to-peer feedback
a routine part of your teaching.
You may also enjoy Gawande's ``\href{http://www.newyorker.com/magazine/2011/10/03/personal-best}{Personal Best}'',
which looks at the value of having a coach.
\end{quotation}   %  class="challenge"
